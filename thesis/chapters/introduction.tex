% Source material: docs/seminar/scientific-narrative.md — Problem Statement, Core Contribution

\chapter{Introduction}
\label{ch:introduction}

Amyotrophic lateral sclerosis (ALS) is a fatal neurodegenerative disease that can be difficult to recognize, especially in its early stages. It targets the motor system in humans, affecting both the upper and lower motor neurons (UMN and LMN, respectively). Although the pathogenesis of ALS is still being investigated, autopsy studies have demonstrated corticospinal tract (CST) degradation in cases of the disease. Diagnosis of ALS is most widely done using the revised El Escorial criteria, which looks for the presence of signs of ALS in both UMN and LMN using clinical examination, electrodiagnostic testing or neuropathological examination, while simultaneously screening for absence of neuroimaging, electrodiagnostic or pathological evidence of other diseases explaining UMN and LMN signs \cite{feldman_amyotrophic_2022}.

Diffusion tensor imaging (DTI) is a promising tool to aid in the diagnosis of ALS. By quantitatively analyzing water diffusion through white matter tracts in the brain, their microstructure and integrity may be estimated \cite{weidman_diffusion_2019}. DTI-derived metrics such as fractional anisotropy (FA) or mean diffusivity (MD) have shown sensitivity to CST degeneration in ALS patients, providing useful clues for the diagnosis of the disease \cite{qin_identifying_2024}.

Currently, a plethora of tools for analysis and visualization of MRI tractograms exist \cite{laamoumi_taxonomic_2025}. However, current CST extraction pipelines using these tools are multi-step, often involving multiple software environments, potentially requiring manual region-of-interest (ROI) placement and producing non-standardized results. There is currently no lightweight, end-to-end, Python CLI tool that standardizes CST extraction with minimal user interaction. This thesis aims to fill this gap with \texttt{csttool}.

\texttt{csttool} is a modular, open-source command-line tool that accepts diffusion-weighted MRI data and produces hemisphere-specific CST bundles along with quantitative tract-level metrics. It implements a six-stage pipeline — environment verification, data import, preprocessing, whole-brain tractography, atlas-based CST extraction, and metric computation — executable as a single command without manual ROI placement. Deterministic tractography was selected over probabilistic methods to prioritize reproducibility and interpretability: by deriving streamlines from principal diffusion directions under fixed parameter settings, the extraction logic remains transparent and stable across subjects. For each hemisphere, \texttt{csttool} computes FA, MD, radial diffusivity (RD), and axial diffusivity (AD) at the tract level, and derives a laterality index to quantify CST asymmetry. The emphasis throughout is on automation, reproducibility, and suitability for clinical single-shell DTI acquisitions.

The rest of this thesis is structured as follows $\dots$
